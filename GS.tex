\section{Geometric Sequences}

\subsection{What is a Geometric Sequence?}

\begin{definition}[Geometric Sequence]
    A sequence $\an$ is said to be \textbf{geometric} if there exists some $r \in \R$ (called the \textbf{common ratio}) such that for all $n \in \N$,
    \begin{align}
        a_{n+1} &= r \cdot a_n
    \end{align}
\end{definition}
In other words, multiplying any term in the sequence by $r$ gives the next term.

\begin{boxexercise}[An Equivalent Characterisation]
    Let $\an$ be a geometric sequence. Show that the $n$th term $a_n$ of the sequence is given by
    \begin{align}
        a_n &= a_1 \cdot r^{n - 1}
    \end{align}
\end{boxexercise}

Observe the following.
\begin{boxhexercise} Let $\an$ be a geometric sequence with common ratio $r$. Show the following:
\begin{enumerate}
    \item If any term is zero (ie, if $a_j = 0$ for some $j \in \N$), then \textit{every} term (except maybe $a_1$) is zero. In particular, every term after the $j$th term is zero.
    \item If $r = 1$, then the sequence is constant (ie, all the terms are equal).
\end{enumerate}
\end{boxhexercise}

Thus, if $\an$ is a geometric sequence with nonzero terms, we can see that for all $n \in \N$,
\begin{align}
    \frac{a_{n+2}}{a_{n+1}} &= \frac{a_{n+1}}{a_n}
\end{align}
Indeed, the ratio of any two successive terms is equal (and indeed, equal to $r$). This is a nice way to check if a sequence is geometric: if you have two pairs of consecutive terms whose ratios are not equal, then the sequence cannot be geometric in nature. Of course, the existence of two pairs of consecutive terms whose ratios are equal does not make the sequence geometric: this would need to be true for \textit{every} pair of consecutive terms.

\subsection{Examples of Geometric Sequences}

Geometric Sequences are as common as arithmetic sequences, and again, we have all seen them in numerous forms, even if we have not always realised it.

\begin{example}
    Each of the following is an example of a geometric sequence.\footnote{As with Arithmetic Sequences, we do not distinguish the sequences (ie, the respective $\N \to \R$ maps) from their terms.}
    \begin{enumerate}[noitemsep]
        \item $3,6,12,24,\ldots$ form a geometric sequence with first term $3$ and common ratio $2$.
        \item $1,3,9,27,\ldots$ (ie, the powers of $3$) form a geometric sequence with first term $1$ and common ratio $3$. (Indeed, the powers of any number $x$ form a geometric sequence with first term $1$, corresponding to the zeroth power, and common ratio $x$.)
        \item $0,0,0,0,\ldots$ form a geometric sequence with first term $0$ and arbitrary common ratio.
        \item For any number $x \in \R$, the sequence $x,x,x,x,\ldots$ is geometric, with first term $x$ and common ratio $1$.
        \item $1,0,0,0,\ldots$ form a geometric sequence with first term $1$ and common ratio $0$.
    \end{enumerate}
\end{example}

\begin{boxhexercise}[The Relationship Between Arithmetic and Geometric Sequences]
\hfill
\begin{enumerate}
    \item Find an example of a sequence $\an$ that is both arithmetic and geometric in nature.
    \item Can you think of a way to convert any arithmetic sequence into a geometric sequence and any geometric sequence into an arithmetic sequence?
    \begin{hint}
        Can you think of a natural way of converting addition into multiplication? That is, can you think of a function $f : \R \to \R$ such that for all $a,b \in \R$, $\fof{a + b} = \fof{a} \fof{b}$? It would then be possible to construct a new sequence from an existing sequence by applying $f$ to every term. Furthermore, if it is possible to invert such an $f$, then it would be possible to convert multiplication back to addition.
    \end{hint}
\end{enumerate}
\end{boxhexercise}

\begin{remark}
    Usually, in the IB, we deal only with ``interesting'' geometric sequences. That is, $r$ is usually taken to not be equal to $0$, $1$ or $-1$, and our sequences are therefore not usually constant/eventually constant. I, however, find these ``exceptional'' cases more interesting...
\end{remark}

In terms of exams, one can expect to see similar types of questions for arithmetic and geometric sequences.
\begin{boxexample}
    Let $\an$ be a geometric sequence with common ratio $r$. Given that $a_2 = 10$ and $a_5 = 80$, find $a_7$.
    \begin{solution}
        Since $a_5$ is $5 - 2 = 3$ terms away from $a_2$, one needs to multiply $a_2$ by $r$ \textit{thrice} to get from $a_2$ to $a_5$. In mathematical terms, we have
        \begin{align*}
            a_5 &= a_2 \cdot r \cdot r \cdot r \\
            \iff r^3 &= \frac{80}{10} = 8 \\
            \ergo \ r &= \cbrt{8} = 2
        \end{align*}
        Therefore, since $a_7$ is $7 - 5 = 2$ terms after $a_5$, we have
        \begin{align*}
            a_7 &= a_5 \cdot r^2 \\
            &= 80 \times 4 = 320
        \end{align*}
        \vspace{-2em}
    \end{solution}
\end{boxexample}

It turns out that things can get slightly more complicated if you are given two terms that are an even number of terms apart.

\begin{boxexample}
    Let $\an$ be a geometric sequence with common ratio $r$. Given that $a_2 = 10$ and $a_6 = 160$, find $a_7$.
    \begin{solution}
        Since $a_6$ is $6 - 2 = 4$ terms away from $a_2$, one needs to multiply $a_2$ by $r$ \textit{four times} to get from $a_2$ to $a_6$. In mathematical terms, we have
        \begin{align*}
            a_6 &= a_2 \cdot r \cdot r \cdot r \cdot r \\
            \iff r^4 &= \frac{160}{10} = 16 \\
            \ergo \ r &= \pm \nthroot{4}{16} = \pm 2
        \end{align*}
        Therefore, $a_7 = a_6 \cdot r = \pm 320$.
    \end{solution}
\end{boxexample}

\begin{boxexercise}
    Let $\an$ be a geometric sequence. In each of the following cases, find the common ratio and the first term.
    \begin{enumerate}
        \item $a_5 = 20$, $a_6 = 100$
        \item $a_3 = 40$, $a_5 = 120$
        \item $a_4 = 40$, $a_6 = 120$
    \end{enumerate}
\end{boxexercise}

\subsection{Sum of $n$ Terms of an Geometric Sequence}

The formula for the sum of $n$ terms of a geometric sequence comes from the following identity.

\begin{boxexercise}
    For all $x \in \R$ and $n \in \N$,
    \begin{align}
        x^{n + 1} - 1 &= \parenth{x - 1}\parenth{1 + x + x^2 + \cdots + x^{n-1} + x^n} \label{eq:id_xnp1-1}
    \end{align}
    \begin{hint}
        Keep it simple. What's the most natural thing to do on the right-hand side?
    \end{hint}
\end{boxexercise}

Notice that the term on the right-hand side of \eqref{eq:id_xnp1-1} is precisely $\parenth{x - 1}$ times the sum of the first $n$ terms of a geometric sequence with first term $1$ and common ratio $x$. So, we can derive the following formula.

\begin{theorem}
    Let $\an$ be a geometric sequence with common ratio $r$. Denote
    \begin{align*}
        S_n &:= \sum_{j=1}^n a_j = a_1 + a_2 + \cdots + a_n
    \end{align*}
    Then, for all $n \in \N$,
    \begin{align}
        S_n &=
        \begin{cases}\displaystyle
            a_1 \cdot \frac{r^n - 1}{r - 1} & \text{ if } r \neq 1 \\
            a_1 \cdot n & \text{ if } r = 1
        \end{cases}
    \end{align}
\end{theorem}
\begin{proof}
    We look at the cases $r = 1$ and $r \neq 1$ separately.
    \begin{description}
        \item[\underline{$r = 1$}.] In this case, the sequence is constant: $a_n = a_1$ for all $n$. Therefore,
        \begin{align*}
            S_n &= \underbrace{a_1 + a_1 + \cdots + a_1}_{n \text{ times}} = a_1n
        \end{align*}
        \item[\underline{$r \neq 1$}.] In this case, we have
        \begin{align*}
            S_n &= a_1 + a_1 r + a_1 r^2 + \cdots + a_1 r^{n-2} + r^{n-1} \\
                &= a_1\parenth{1 + r + r^2 + \cdots + r^{n-2} + r^{n-1}} \\
                &= a_1 \parenth{\frac{r^n - 1}{r - 1}} \quad \quad \text{(by rearranging \eqref{eq:id_xnp1-1})}
        \end{align*}
    \end{description}
    We therefore have the desired result.
\end{proof}

As with arithmetic sequences, the key to solving problems involving geometric sequences is \textbf{practice}. To that end, here are a few practice exercises.

\begin{boxexercise}
    Let $\an$ be a geometric sequence. In each of the following cases, find the sum of the first $n$ terms.
    \begin{enumerate}[noitemsep]
        \item $a_1 = 20$, $r = 2$, $n = 10$
        \item $a_1 = 3$, $a_6 = \frac{81}{8}$, $n = 6$
        \item $a_2 = 5$, $a_{4} = 125$, $n = 3$
    \end{enumerate}
    \begin{hint}
        There can be more than one solution!
    \end{hint}
\end{boxexercise}

