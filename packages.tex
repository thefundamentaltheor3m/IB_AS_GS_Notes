\usepackage[utf8]{inputenc}
\usepackage[english]{babel}


\usepackage{tcolorbox}
\usepackage{authblk}  % Lets you add an \affil{} to your title, stating your affiliation {eg. Sigma Mathematics Society}
\usepackage{ragged2e}
\usepackage{csquotes}
\usepackage{pdfpages}

\usepackage[a4paper, total={6.6in, 9in}, textheight=720pt, voffset=35pt, footnotesep=25pt]{geometry} % Adjust margins as necessary

\usepackage{xfrac}
\usepackage{cancel}

\usepackage[inline]{enumitem}

%\usepackage{tgpagella}

\usepackage{blindtext}
\usepackage{lipsum}
\usepackage{verbatim}
\usepackage{hyperref}
\hypersetup{
    citebordercolor = 1 1 1,
    linkbordercolor = 1 1 1,
    filebordercolor = 1 1 1,
    menubordercolor = 1 1 1,
    urlbordercolor = 1 1 1,
    colorlinks  =   true,
    linkcolor   =   blue,
    citecolor   =   magenta,
    urlcolor    =   blue
}

\usepackage{biblatex} % Modify citation format using [style=yourstyle] parameter--eg \usepackage[style=mla-new]{biblatex}
\bibliography{References.bib}
\addbibresource{References.bib}

\usepackage{cancel}
\usepackage{amssymb}
\usepackage{amsmath}
\setcounter{MaxMatrixCols}{30}
%\usepackage{MnSymbol}
\usepackage{mathrsfs}
\usepackage{mathtools}

\usepackage{array}
\usepackage{booktabs}
\usepackage{longtable}

\usepackage{graphicx}
\usepackage{wrapfig}
\usepackage{caption}
\usepackage{subcaption}
\usepackage{tikz}
\usepackage{xcolor}

% Theorem layout

\usepackage{amsthm}

\theoremstyle{definition}
\newtheorem{definition}{Definition}[section]

\newtheorem*{theorem*}{Theorem}
\newtheorem{theorem}{Theorem}[section]
\newtheorem{corollary}{Corollary}[theorem]
\newtheorem{lemma}[theorem]{Lemma}
\newtheorem{conjecture}[theorem]{Conjecture}
\newtheorem{algorithm}[theorem]{Algorithm}
\newtheorem{proposition}[theorem]{Proposition}
\newtheorem{example}[theorem]{Example}
\newtheorem{wexample}[theorem]{Worked Example}
\newtheorem{exercise}[theorem]{Exercise}
\newtheorem{hexercise}[theorem]{Challenge Problem}
\newtheorem*{claim*}{Claim}
\newenvironment{boxexample}{
    \begin{tcolorbox}[colback=blue!5!white,colframe=blue!75!black]\begin{wexample}
}{
    \end{wexample}\end{tcolorbox}
}
\newenvironment{boxexercise}{
    \begin{tcolorbox}[colback=yellow!5!white,colframe=yellow!50!black]\begin{exercise}
}{
    \end{exercise}\end{tcolorbox}
} % colback=red!5!white,colframe=red!75!black
\newenvironment{boxhexercise}{
    \begin{tcolorbox}[colback=red!5!white,colframe=red!75!black]\begin{hexercise}
}{
    \end{hexercise}\end{tcolorbox}
}
\newenvironment{solution}{\begin{proof}[Solution]\renewcommand{\qedsymbol}{}}{\end{proof}}

\newtheoremstyle{hint}
    {\dimexpr\topsep/2\relax} % space above
    {\dimexpr\topsep/2\relax} % space below
    {\it}          % body font
    {}          % indent amount
    {\it} % theorem head font
    {:}         % punctuation after theorem head
    {.5em}      % space after theorem head
    {}          % theorem hed spec. (empty = "normal")

\theoremstyle{hint}
\newtheorem*{remark}{Remark}
\newtheorem*{hint}{Hint}

% Structure and Numbering

\usepackage{fancyhdr}
\usepackage{lastpage}

\pagestyle{fancy}
\fancyhf{}

\rhead{{Page \thepage }}%\hspace{1pt} of \pageref{LastPage}}}
\lhead{\textit\slshape\nouppercase{\leftmark}}

\numberwithin{equation}{section}

% Spacing and indentation

\usepackage{setspace}
\renewcommand{\baselinestretch}{1.5}

\usepackage[skip=18pt, indent=0pt]{parskip}
\usepackage{indentfirst}