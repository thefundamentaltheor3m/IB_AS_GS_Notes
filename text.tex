\section{Sequences: A Brief Introduction}

Generally speaking, we can think of a (real) sequence as being an infinite list of real numbers that can be indexed by the natural numbers. That is, a sequence has a well-defined first element, second element, third element, and so on.

It is important to note that certain things cannot be ``sequenced": for instance, a famous argument known as \textbf{Cantor's Diagonal Argument} shows that you cannot fit every single real number into a sequence. Indeed, it isn't too difficult to see intuitively that if you look at the interval $\brac{0,1}$, you cannot possibly list down every single element. You can think of this as being due to the fact that you can have as many decimal places as you want, for example.

Anyway, the bottom line is, a sequence is a list of numbers as well as a well-defined way of enumerating them. We usually denote sequences by something like
\begin{align*}
    \parenth{a_n}_{n=1}^{\infty}
\end{align*}
consisting of elements $a_1, a_2, a_3, a_4, \ldots$ which all stand for real numbers. The letter $a$ is used to identify the sequence---that is, it is the ``name" of the sequence---and the subscripts, also called indices, are used to denote various numbers in the sequence. Formally speaking, one can think of a sequence as a function taking in natural numbers (ie, some element of $\N = \set{1, 2, 3, \cdots}$) and outputting real numbers. So, for every index $n$, there is some real number $a_n$ associated with the sequence $a$.

\section{Arithmetic Sequences}

\subsection{What is an Arithmetic Sequence?}

\begin{definition}[Arithmetic Sequence]
    We say that a sequence $\parenth{a_n}_{n=1}^{\infty}$ is an \textbf{arithmetic sequence} if there is some number $d$, known as the \textbf{common difference}, such that for all $n \in \N$,
    \begin{align}
        a_{n+1} &= a_n + d
    \end{align}
    In other words, adding $d$ to any term in the sequence gives the next term.
\end{definition}

\begin{boxexercise}[An Equivalent Characterisation]
Let $\an$ be an arithmetic sequence. Show that the $n$th term $a_n$ of the sequence is given by
\begin{align}
    a_n = a_1 + d \cdot \parenth{n - 1} \label{eq:AS_nth_term}
\end{align}
\end{boxexercise}

\subsection{Examples of Arithmetic Sequences}

Arithmetic sequences appear everywhere, and we are already used to working with them, even if we may not realise it.

\begin{example}
We already know of the following arithmetic sequences:
\begin{enumerate}[noitemsep]
    \item The sequence of natural numbers $1,2,3,4,5,6,\ldots$ form an arithmetic sequence with first term $1$ and common difference $1$.\footnote{Formally speaking, this isn't quite true: the sequence itself is the \textit{map} $a_n = n$ that corresponds to the trivial inclusion map from $\N$ to $\R$ that sends every element of $\N$ to itself. We have essentially identified each sequence with its elements, as we are more interested in those than the maps themselves.}
    \item The two-times-table $2, 4, 6, 8, 10, 12, \ldots$ is an arithmetic sequence with first term $2$ and common difference $2$. Indeed, an analogous result is true for \textit{any} multiplication table.
    \item The odd numbers $1, 3, 5, 7, 9, 11, \ldots$ is also an arithmetic sequence with first term $1$ and common difference $2$.
\end{enumerate}
\end{example}

Given two terms of an arithmetic sequence, it is easy to compute the common difference $d$ by looking at the \textbf{number of differences} separating the two terms. For example:

\begin{boxexample}
Let $\an$ be an arithmetic sequence with $a_2 = 5$ and $a_5 = 14$. Find the common difference of the sequence.
\begin{solution}
Since $a_2$ and $a_5$ are $5 - 2 = 3$ terms apart, adding $d$ to $a_2$ three times should give $a_5$. In other words,
\begin{align*}
    5 + 3d &= 14 \\
    \iff \quad \quad \quad \quad d &= \frac{14 - 5}{3} \\
    &= 3
\end{align*}
\end{solution}
\vspace{-3em}
\end{boxexample}

Try the following yourself.

\begin{boxexercise}
Let $\an$ be an arithmetic sequence. In each of the following cases, find the common difference and the first term.
\begin{enumerate}[noitemsep]
    \item $a_5 = 10$, $a_{45} = 90$
    \item $a_3 = 2$, $a_7 = 4$
    \item $a_{365} = 364$, $a_{1001001} = 1001000$
\end{enumerate}
\end{boxexercise}

\subsection{Sum of $n$ Terms of an Arithmetic Sequence}

It turns out that there is a rather ingenious trick to sum the first $n$ terms of an arithmetic sequence. Consider the following example to compute the first $100$ terms of $1, 2, 3, 4, \ldots$ (ie, the natural numbers), an arithmetic sequence with first term $1$ and difference $1$.

\begin{boxexample}[Gauss] \label{eg:Gauss_sum}
    It is said that the renowned German mathematician, Carl Friedrich Gauss, came up with the following trick as a child when his teacher asked him to add the first $100$ natural numbers as a punishment for misbehaving in class. \\

    He first wrote down the numbers like so:
    \begin{align*}
        \begin{matrix}
            S & = & 1 & + & 2 & + & 3 & + & \cdots & + & 98 & + & 99 & + & 100
        \end{matrix}
    \end{align*}
    Then, he wrote another copy below, like so:
    \begin{align*}
        \begin{matrix}
            S & = & 1 & + & 2 & + & 3 & + & \cdots & + & 98 & + & 99 & + & 100 \\
            S & = & 100 & + & 99 & + & 98 & + & \cdots & + & 3 & + & 2 & + & 1
        \end{matrix}
    \end{align*}
    Then, he added the numbers column-wise:
    \begin{align*}
        \begin{matrix}
            S & = & 1 & + & 2 & + & 3 & + & \cdots & + & 98 & + & 99 & + & 100 \\
            S & = & 100 & + & 99 & + & 98 & + & \cdots & + & 3 & + & 2 & + & 1 \\
            \hline
            2S & = & 101 & + & 101 & + & 101 & + & \cdots & + & 101 & + & 101 & + & 101
        \end{matrix}
    \end{align*}
    In other words, $2S = 101 \times 100$, meaning $S = 5050$.
\end{boxexample}

The heart and soul of Gauss's argument is the fact that these ``complementary terms'' $1$ and $100$, $2$ and $99$, and so on---that is, $j$ and $n - j + 1$ for every $j \in \N$ with $1 \leq j \leq 100$---add up to the same number, namely $101$. It turns out that a similar result is true of \textit{all} arithmetic sequences, allowing us to generalise Gauss's argument.

\begin{boxexercise}\label{ex:AS_first_pl_last}
    Let $\an$ be an arithmetic sequence and let $n$ be any natural number. Show that for any $j \in \N$ such that $1 \leq j \leq n$,
    \begin{align}
        a_j + a_{n-j+1} &= a_1 + a_n \label{eq:AS_first_pl_last}
    \end{align}

    \begin{remark}
        While this is easily proved by substituting appropriately into the formula \eqref{eq:AS_nth_term}, it is instructive to instead think about the equivalent statement
    \begin{align*}
        \frac{a_j + a_{n-j+1}}{2} &= \frac{a_1 + a_n}{2} \label{eq:AS_first_pl_last}
    \end{align*}
    This result is obvious when $a_n = n$, ie, when you just think about the $j$th and $\parenth{n-j+1}$th natural numbers instead of those terms in the sequence. It is precisely the arithmeticity of the sequence that allows us to use the natural number formulation and the sequence formulation interchangeably---in this case, at least.
    \end{remark}
\end{boxexercise}

We therefore have the following result:
\begin{theorem}
Let $\an$ be an arithmetic sequence. Let
\begin{align}
    S_n &:= \sum_{j=1}^{n} a_j
\end{align}
denote the sum of the first $n$ terms of $\an$---in other words, $S_n = a_1 + a_2 + \cdots + a_n$. Then, we have
\begin{align}
    S_n &= \frac{n}{2}\parenth{a_1 + a_n} \label{eq:AS_sum_form}
\end{align}
\end{theorem}
\begin{proof}
    We use the same argument as Example \ref{eg:Gauss_sum}:
    \begin{align*}
    \begin{matrix}
        S_n & = & a_1 & + & a_2 & + & \cdots &  + & a_{n-1} & + & a_n \\
        S_n & = & a_n & + & a_{n-1} & + & \cdots & + & a_2 & + & a_1 \\
        \hline
        2S_n & = & \parenth{a_1 + a_n} & + & \parenth{a_1 + a_n} & + & \cdots & + & \parenth{a_1 + a_n} & + & \parenth{a_1 + a_n}
    \end{matrix}
    \end{align*}
    where the last line follows from Exercise \ref{ex:AS_first_pl_last}.
    
    Therefore,
    \begin{align*}
        2S_n &= n\parenth{a_1 + a_n} \\
        \iff \quad \quad S_n &= \frac{n}{2}\parenth{a_1 + a_n}
    \end{align*}
    as required.
\end{proof}
Hereafter, we will discard the summation notation and instead use $S_n$ to denote the sum of the first $n$ terms of a given sequence.

Sometimes, we do not know what the $n$th term of some sequence is. Therefore, the following equivalent formula, which is easily derived, is quite useful.
\begin{boxexercise}[An Equivalent Formula]
Let $\an$ be an arithmetic sequence with common difference $d$. Then,
\begin{align}
    S_n &= \frac{n}{2}\parenth{2a_1 + d\parenth{n-1}}
\end{align}
\end{boxexercise}
Once one gets more comfortable with arithmetic sequences, one will start seeing patterns more intuitively, and be able to add up terms more easily. The key for this is, of course, \textbf{practice}.
\begin{boxexercise}
    Let $\an$ be an arithmetic sequence. In each of the following cases, find the sum of the first $n$ terms.
    \begin{enumerate}[noitemsep]
        \item $a_1 = 20$, $d = 3$, $n = 10$
        \item $a_1 = 3$, $a_6 = 101$, $n = 6$
        \item $a_5 = 49$, $a_{7} = 51$, $n = 11$
        \item $a_4 = 14$, $a_6 = 22$, $n = 15$
    \end{enumerate}
    \begin{hint}
        The formula \eqref{eq:AS_sum_form} can be rewritten as
        \begin{align*}
            S_n &= n\parenth{\frac{a_1 + a_n}{2}}   
        \end{align*}
        Is this a more helpful form? Think about the quantity in parentheses.
    \end{hint}
\end{boxexercise}

We are now ready to move onto Geometric Sequences.
\section{Geometric Sequences}

\subsection{What is a Geometric Sequence?}

\begin{definition}[Geometric Sequence]
    A sequence $\an$ is said to be \textbf{geometric} if there exists some $r \in \R$ (called the \textbf{common ratio}) such that for all $n \in \N$,
    \begin{align}
        a_{n+1} &= r \cdot a_n
    \end{align}
\end{definition}
In other words, multiplying any term in the sequence by $r$ gives the next term.

\begin{boxexercise}[An Equivalent Characterisation]
    Let $\an$ be a geometric sequence. Show that the $n$th term $a_n$ of the sequence is given by
    \begin{align}
        a_n &= a_1 \cdot r^{n - 1}
    \end{align}
\end{boxexercise}

Observe the following.
\begin{boxhexercise} Let $\an$ be a geometric sequence with common ratio $r$. Show the following:
\begin{enumerate}
    \item If any term is zero (ie, if $a_j = 0$ for some $j \in \N$), then \textit{every} term (except maybe $a_1$) is zero. In particular, every term after the $j$th term is zero.
    \item If $r = 1$, then the sequence is constant (ie, all the terms are equal).
\end{enumerate}
\end{boxhexercise}

Thus, if $\an$ is a geometric sequence with nonzero terms, we can see that for all $n \in \N$,
\begin{align}
    \frac{a_{n+2}}{a_{n+1}} &= \frac{a_{n+1}}{a_n}
\end{align}
Indeed, the ratio of any two successive terms is equal (and indeed, equal to $r$). This is a nice way to check if a sequence is geometric: if you have two pairs of consecutive terms whose ratios are not equal, then the sequence cannot be geometric in nature.

\subsection{Examples of Geometric Sequences}

Geometric Sequences are as common as arithmetic sequences, and again, we have all seen them in numerous forms, even if we have not always realised it.

\begin{example}
    Each of the following is an example of a geometric sequence.\footnote{As with Arithmetic Sequences, we do not distinguish the sequences (ie, the respective $\N \to \R$ maps) from their terms.}
    \begin{enumerate}[noitemsep]
        \item $3,6,12,24,\ldots$ form a geometric sequence with first term $3$ and common ratio $2$.
        \item $1,3,9,27,\ldots$ (ie, the powers of $3$) form a geometric sequence with first term $1$ and common ratio $3$. (Indeed, the powers of any number $x$ form a geometric sequence with first term $1$, corresponding to the zeroth power, and common ratio $x$.)
        \item $0,0,0,0,\ldots$ form a geometric sequence with first term $0$ and arbitrary common ratio.
        \item For any number $x \in \R$, the sequence $x,x,x,x,\ldots$ is geometric, with first term $x$ and common ratio $1$.
        \item $1,0,0,0,\ldots$ form a geometric sequence with first term $1$ and common ratio $0$.
    \end{enumerate}
\end{example}

\begin{boxhexercise}[The Relationship Between Arithmetic and Geometric Sequences]
\hfill
\begin{enumerate}
    \item Find an example of a sequence $\an$ that is both arithmetic and geometric in nature.
    \item Can you think of a way to convert any arithmetic sequence into a geometric sequence and any geometric sequence into an arithmetic sequence?
    \begin{hint}
        Can you think of a natural way of converting addition into multiplication? That is, can you think of a function $f : \R \to \R$ such that for all $a,b \in \R$, $\fof{a + b} = \fof{a} \fof{b}$? It would then be possible to construct a new sequence from an existing sequence by applying $f$ to every term. Furthermore, if it is possible to invert such an $f$, then it would be possible to convert multiplication back to addition.
    \end{hint}
\end{enumerate}
\end{boxhexercise}

\begin{remark}
    Usually, in the IB, we deal only with ``interesting'' geometric sequences. That is, $r$ is usually taken to not be equal to $0$, $1$ or $-1$. However, personally, I find these cases to be more interesting...
\end{remark}

In terms of exams, one can expect to see similar types of questions for arithmetic and geometric sequences.
\begin{boxexample}
    Let $\an$ be a geometric sequence with common ratio $r$. Given that $a_2 = 10$ and $a_5 = 80$, find $a_7$.
    \begin{solution}
        Since $a_5$ is $5 - 2 = 3$ terms away from $a_2$, one needs to multiply $a_2$ by $r$ \textit{thrice} to get from $a_2$ to $a_5$. In mathematical terms, we have
        \begin{align*}
            && a_5 &= a_2 \cdot r \cdot r \cdot r &&& \\
            &\iff& r^3 &= \frac{80}{10} = 8 &&& \\
            &\ergo& r &= \cbrt{8} = 2
        \end{align*}
        Therefore, since $a_7$ is $7 - 5 = 2$ terms after $a_5$, we have
        \begin{align*}
            a_7 &= a_5 \cdot r^2 \\
            &= 80 \times 4 = 320
        \end{align*}
        \vspace{-2em}
    \end{solution}
\end{boxexample}

It turns out that things can get slightly more complicated if you are given two terms that are an even number of terms apart.

\begin{boxexercise}
    Let $\an$ be a geometric sequence. In each of the following cases, find the common ratio and the first term.
    \begin{enumerate}
        \item 
    \end{enumerate}
\end{boxexercise}